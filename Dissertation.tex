\documentclass[11pt]{article}
\usepackage[margin=1in]{geometry}
\usepackage{amsfonts, amsmath, amssymb}
\usepackage[none]{hyphenat}
\usepackage{fancyhdr}
\usepackage{graphicx}
\usepackage{float}
\usepackage[nottoc, notlot, notlof]{tocbibind}

\pagestyle{fancy}
\fancyfoot{}
\fancyfoot[C]{\thepage}
\renewcommand{\footrulewidth}{0pt}
\parindent 3ex
\setlength{\parindent}{4em}
\setlength{\parskip}{1em}
\renewcommand{\baselinestretch}{1.5}

\begin{document}
\begin{titlepage}
\begin{center}
\vspace*{1cm}
\vfill
\huge{\textbf{Trust and Macroeconomic development}}\\[3mm]
\vfill
Hannah Lee\\
\today\\
\end{center}
\end{titlepage}

\tableofcontents
\thispagestyle{empty}
\clearpage
\setcounter{page}{1}

\section{Introduction}
\subsection{Background}

Economic theory suggests that market failure arise when contracts are difficult to enforce or observe. Social capital can help these failures. The more individuals trust each other, the more they are able to contract with each other. Hence many believe that trust is a critical input for both macro- and micro outcomes.

The intellectual tradition stressing the importance of social attitudes date back to at least Max Weber, and has seen many recent restatements in political science in particular, recent ones being Robert Putnam showed, for instance, how civic attitudes and trust could account for differences in economic and government performance between northern and southern Italy. This tradition has been so influential that it has led the World Bank to put on the top of the agenda the promotion of a new kind of capital stock to trigger economic development: the social capital. Following Fukuyama, this social capital can be defined as the "set of informal values or norm shared among members of a group that permits them to cooperate with one another". Obviously the propensity to trust each other is likely to be key for fostering such mutual cooperation and growth.

But paradoxically, economists are still struggling to providing empirical evidence on such a causal impact of social attitudes on economic development. Most of the time, the economic literature has been successful to emphasise the existence of a cross-country correlation, instead of a causal relationship, between growth and social attitudes. For instance, Stephen Knack and Philip Keefer, economists at the World Bank, proposed to measure the cross-country heterogeneity in social capital by using an international social survey(the World Values Survey), which reports direct information on the level of trust of people asking them: "Generally speaking, would you say that most people can be trusted, or that you need to be very careful in dealing with people?" Knack and Keefer then showed a strong cross-country correlation pattern between country levels of trust and country levels of income per capita.

Yet this correlation pattern leaves completely unexplained the relation of causality between trust and growth and may result in misguided policy recommendations. Individuals who are living in wealthy countries, with efficient institutions, are likely to more trusting people than individuals living in developing countries at war. The relation between trust and growth could thus go the other way around, in which case a prerequisite to strengthen social cooperation is to foster economic development first.

What do economists need to uncover a causal relationship running from social attitudes toward economic development? They basically need to find social attitudes that not over-determined by economic development. A priori, this is a difficult task to the extent that social attitudes can always been influenced by economic and social environment where people live. These difficulties show up even in the most complete and remarkable pieces of work of Guido Tabllini, who shows that the literacy rate and the political institutions in place over the past several centuries are correlated with trust at the end of the twentieth century, and thus capture current impact of trust on the income per capital across European regions. But since these historical variables are time-invariant, they could also pick up the more fundamental influence of specific time-invariant features such as legal origins, political institutions, and more generally historical and geographical factors. At the end of the day, the question whether it is trust or any other specific feature which matters for explaining growth is thus still open.

\subsection{Research Aim}

This paper proposes a new method to uncover the causal link between trust and economic development by using Agent-Based modelling approach. Its empirical strategy is based on the cogent definitions and dynamics of trust of the agents in the model. In the spirit of the epidemiological literature, it weeks to finding out the qualitative/quantitative effects of trust on economic growth in a simulated model. That is, understanding how computational trust models help to achieve economic interaction providing large-scale, virtual social networks.


\section{Literature Review}
\subsection{Understanding Trust through lens of Economics}
\subsubsection{Notion of Trust in Economics}

The concept of trust is prevalent in our society. From medical treatment to business trade, we must establish certain level of trust before acting. In the domain of Economics, trust management plays an important role in decision making.  As stressed by Kenneth Arrow:1 “Virtually every commercial transaction has within itself an element of trust, certainly any transaction conducted over a period of time. It can be plausibly argued that much of the economic backwardness in the world can be explained by the lack of mutual confidence.”

However trust is a complex concept and has several connotations(Mcknight and Chervany 1996). According to Gambetta (1988)'s definition, Trust(or symmetrically distrust) is a particular level of the subjective probability in with which an agent assesses that another agent or group of agents will perform a particular action, both before it can monitor such action(or independently of his capacity ever to be able to monitor or enforce it) and in a context in which it affects his own action. From this definition we can see that trust is (1)subjective, because it is estimated from the perspective of the individual trustor. Furthermore, trust exists before the action is monitoered. That is to say, trust is a (2)prior belief about an agent's behavior. Another point needs to be noticed is that trust is in a (3)certain context, which affects trustor's own actions.

On the basis of conceptual disambiguition of analytic model, Castelfranchi(2010)distinguishes trust from other related concepts, and categorize trust in four parts. (Trust Theory chapter8 ... )

In sum, these factors are combined to present a single trust metric T for peer u:
$T(u) = alpha \sum_{i=1}^{I(u)} S(u,i) Cr(P(u,j)) TF(u,i) + beta CF(u)$
where I(u) denotes the number of transactions of peer u with other peers and p(u,i) denotes the interaction partner(peer) in peer u's ith transaction. S(u,i) is the normalized amount of positive ratings u receives from p(u,i) in ith transaction. Cr(v) denotes the credibility of the feedback provided by peer v, TF(u,i) is the adapative transaction factor for u's ith transaction, and CF(u) is the adaptive community context factor for peer u, alpha, and beta are the normalized weights for the collective evaluation and the community context factor.

In order to address this issue, a lot of approaches have been proposed(Buchegger and Boudec 2003; Xiong and Liu 2004; Teacy et al. 2006; Lauw et al. 2006; Zhang and Cohen 2008) and a comparison study of social network-based  and probabilistic estimation-based approaches is conducted(Dspotovic and Aberer 2006). Among all, game theoretical approaches(Zacharia 1999; Schillo et al. 2000) to predict behvaior of the interaction of game partner. (Game thoeretic approaches has its own value... but)
  
What the trust game measures is an ongoing controversy among many economists. Early studies showed that the trust game does not measure trust at all, but rather trustworthiness(Glaeser et al.(2000)). However more recent research suggests that the trust game measures the beliefs component of trust(how likely is it that an unknown other will cheet you). See, e.g., Sapienza, Toldra and Zingales(2012); or Butler, Gulian and Guiso(2012). This raises the important pratical concern over whether behvaiors in the widely-used "trust game" actually measure trust, or instead reveal more about risk attitudes. It is critical to confront this question rigorously, as data from these games are increasingly used to support conclusions from a wide variety of fields including macroeconomic development.


\subsection{Agent-Based Modelling approach and NetLogo}
\subsubsection{Motive for ABM approach}

(ABM introduction)

Intuitively, direct experience survey(e.g., WVS) is the most reliable and personalized information for trust assessment. Nevertheless, if we want to undertake problems found in socially complex virtual society, like negotiation issues, more sophisticated models are needed. That is because in large-scale, open systems like in real society's direct experience is often not sufficient or even non-existent.

One of the advantages of convoluted modelling like Agent based modelling is that it could fill the gap between those logics. Since structure of the mental state is reflected in the model, which can be as important as the final value, processes like argumentation, automated behvior can take place.

Furthremore, in an open, complex system, it is nontrivial to understand and aggregate such information due to its uncertainty, and ABM can deal with this problem effectively. For example, information reporter may provide counterfeit opinions, or although the provided information is correct, it may not be suitable for the information requester due to its personalized view of the system. In a large-scale social network system like this the performance of game theoretical models decreases due to high complexity of relations and interactions among agents. Rather, probabilistic estimation-based approaches can then be conducted to fill the gap.

We can gain knowledge principally upon the laboratory since ethics, legality, and costs are factors that most field experiments infeasible. For example, one hopes that Federal Reserve Chair Janet Yellen does not plan on deploying field experiments to refine her knowledge of the effect of interest rate on inflation. Similarly, field experiments on the effects of trust are impractical in real world. In such cases, laboratory experiments like this often provide the best(albeit impoerfect) source of information on causal effects.


Another advantage is the proximity with human comprehension, for a huma being, it is easier to understand an explanation based on beliefs, desires and intentions than an explanation full of formulas.    

\subsubsection{Computation Methods}

Generally trust models could be divided into sub categories by which approach the trust is calculated. We next review some representative trust computation methods.

(1) Summation / Average
The simplest approach to compute trust is to simply aggregate all the ratings. There are several types of aggregations. For instance, in eBay's reputation system(http://www.ebay.com), for recent 1, 6, 12 months, positive, neutral and negative feedback are summed separately.(Fig1)

(2) Bayesian Inference
The beta reputation system proposed by Josang and Ismail (2002) estimates reputation of an agent using a probabilistic model, i.e., beta probability density function. The beta distributions are a family of statistical distribution functions that are characterized by two parameters $alpha$ and $beta$. The beta probability density function is defined as follows:
$$\beta(p|alpha, beta)=frac{Gamma(alpha + beta}{Gamma(alpha)\Gamma(beta)}p^{alpha-1}(1-p)^{beta-1}$$
where p[0,1] is a probability variable, $alpha$ , $beta$ > 0. This function shows the relative likelihood of the values for p, given the parameters $alpha$ and $beta$. The probability expectation of the beta distribution is computed by E(p)=$alpha$ / ($alpha$ + $beta$). In the beginning, when knowledge is not available, the priori distribution is actually the uniform beta PDF with the parameter $alpha$ = 1 and $beta$ = 1.(Fig2) After observing $x$ positive and $y$ negative outcomes, the posterior distribution becomes another beta PDF with $alpha$ = 1 + $x$ and $beta$ = 1 + $y$ . Figure 2 gives an example when $x$ = 10 and $y$ = 2. Following this model, we assume the feedback collected from other agents who have interacted with the target agent is binary(1 or 0), i.e., 1 means that the target agent has a good reputation and 0 otherwise. These third-party opinions are combined by simply accumulating the number of positive feedback(a) and the number of negative feedback(b). Hence to make $alpha$ , $beta$ >0, their values are set as $alpha$ = 1 + a and $beta$ = 1 + b, respectively.

(3) Iterative Methods
Another class of methods compute trust through transitive iteration. A representative trust model using such a computation method is EigenTrust(Kamvar et al. 2003), which is a reputation system developed for Peer-to-Peer networks. EigenTrust tries to fulfill the goals of self-policing, no profit for newcomers, anonymity-maintaining, minimal overhead, and robust to malicious collective of peers.

EigenTrust calculates a global reputation for each peer in the network based on the local opinions of all other peers. Every peer $i$ stores for every interaction partner $j$ a trust value $s_{ij}$, representing the experience it has gained. $s_{ij}$ is normalized locally(denoted by $c_{ij}$) to avoid the arbitrary values assigned by malicious peers. Peer $i$ then asks its acquaintance what they think about other peers and receives the the normalized trust values. All the normalized values are stored in a matrix. From this matrix, peer $i$ can retrieve any peer $j$'s global reputation $t_{ij}$ by calculating 
% $$t_{ij} = sum_{k=0} c_{ik}\c_{kj}$$
So peer's global reputation is the weighted sum of all other peers' opinion where the weight factor is global reputation of the opinion reporter.(examples...)

\subsubsection{Introduction to NetLogo}

NetLogo is a programming environment which allows for the construction and exploration of agent-based models. It proposes a particularly useful functionality for the manipulation of agent-based models. It allows us to give instructions to groups or subgroups of agents, or agent sets. As a result, it is possible to collectively control all of the turtles as well as to pick a smaller subset and give it instructions that are not followed by the remaining agents. (So as to do this, it is possible to create species or breeds which can be manipulated as groups of agents).

The NetLogo platform corresponds to a simulation approach said to be "in time-discrete intervals", which means that it makes a collective group of entities evolve in successive time intervals of equal length. (It is vital to understand that as the models run as a time-series, the value of ouptut variables change, and this may occur because of autocorrelation) The corresponding modelling approach therefore consists of identifying the behavior of each one across each time interval. This approach is centered on entities involved, otherwise known as agents. NetLogo's metamodel identifies three different types of entities which can be modeled 1) the environment: this is a rectangular space modeled in the form of a regular grid of nxm suqare tiles; 2)the mobile agents: these move within the environment and interact with it and each other; 3)the links: these are dynamically created inbetween agents.

Each agent is aware of its own state. An agent is said to be autonomous, in the sense that it can decide what to do based on endogenous goals and information, much like a social actor, without necessarily requiring exogeneous guidance. Besides making decisions based on its own internal state, an agent can also decide to act in reaction to some perceived enviornmental situation. 

NetLogo can be adapted to allow the observer greater ability to manipulate independent agents, or to create a model closer to reality. An observer may wish to include procedures that allow for the level of trust be increased or reduced. This procedure could be helpful for determining how the agent's microscopic behaviors are evolve into macroscopic pattern. Since agents can also behave proactively, based on goals agents can communicate, sometimes generating emergent patterns of sociality(e.g., collective behavior) by making their attributes visible or actually passing information. This is important traits because heterogeniety in trust beliefs coupled with tendency of indiciduals to extrapolate beliefs about others from their own level of trustworthiness could generate non-monotonic relationship.

\section{Model Description}
\subsection{Overview of Simulation Step}

Economists have singled out trust as an important phenomenon at the individual level, and for society as a whole. This raises natural questions about the determinants of trust. We knew already that decisions about whom and how much to trust are partly based on pecuniary considerations, but that there is a substantial and varied role for moral considerations which many of the economists have missed.

In this paper we suggest that receiver's belief in what the senders would consider as trust is strongly correlated with how receivers themselves define trust. This is a pattern consistent with the psychological phenomenon of false consensus(Ross et al. 1977) in which we tend to think that others are like us. And this idea is reflected in the Iterative Methods mentioned earlier.(...)

Also if some individual hold persistently mistaken trust beliefs, then these mistaken beliefs may create substantial economic losses. 

  

\subsection{Mathematical Representation}
\subsection{Validation}

If the intention is to accurately represent economic reality, then validation is about assessing how well the model is capturing the quintessence of its empirical referent. This might be measure in terms of goodness of fit to the characteristics of the models' referent.
 


\section{Experiments and Results}
\subsection{Scenario Analysis}

\subsection{Results}

\section{Conclusions}

We document both excessive trust and excessive mistrust are not only individually costly, but also incur social cost. This is accordance with the experimental data suggesting the cost of trusting too much. It is because from societal view, (...)

Obviously, there are drawbacks in modelling the cognition of human being. Because of their complexity, some of the models remain at a descriptive level. 

\pagebreak
\begin{thebibliography}{}


\bibitem{name1} %journal article
Rafa Bbaptista, J Doyne Farmer, Marc Hinterschweiger, Katie Low, Daniel Tang and Arzu Uluc,
''Macroprudential policy in an agent-based model of the UK housing market"
Staff Working Paper No. 619(2016)
Bank of England

\bibitem{name2} %book
Uri Wilensky and William Rand,
\textit{An Introduction to Agent-Based Modeling - Modeling Natural, Social and Engineered Complex Systems with NetLogo}.
Cambridge, Massachusetts,
2015.
The MIT Press,

\bibitem{name3} %book
Lynne Hamill and Nigel Gilbert,
\textit{Agent-based modelling in economics}.
Chichester, UK,
Centre for Research in Social Simulation (CRESS),
2015.
Wiley

\bibitem{name4} %book
Steven F. Railsback and Volker Grimm,
\textit{Agent-based and individual-based modeling : a practical introduction}.
Princeton, NJ,
2012.
Princeton University Press

\bibitem{name5} %book
Nigel Gilbert, Klaus G. Troitzsch,
\textit{Simulation for the social scientist}.
Philadelphia, Pa,
1999.
Open University Press

\bibitem{name6} %book
Romulus-Catalin Damaceanu,
\textit{Nonlinear dynamic modeling of economic systems using NetLogo}.
Saarbrucken,
2014.
LAP Lambert Academic Publishing

\bibitem{name7} %book
Joshua M. Epstein,
\textit{Agent zero : toward neurocognitive foundations for generative social science}.
Princeton, NJ,
2013.
Princeton University Press

\bibitem{name8} %website
NorthWestern University NetLogo Team,
\textit{NetLogo Model Library}
1/5/2017.
\texttt{<http://ccl.northwestern.edu/netlogo>.}

\bibitem{name9} %website
Economic Simulation Library(ESL).
\texttt{Standalone model of the UK housing market}
1/5/2017.
\texttt{<https://github.com/EconomicSL/housing-model>.}

\bibitem{name10} %journal article
Daniel Houser, Daniel Schunk and Joachim Winter,
''Trust Game measures trust"
Sonderforschungsbreich 504 06-14(2006)
Manheim Research Institute for the Economics of Aging

\bibitem{name11} %journal article
Steven D. Levitt, John A. List,
''”Field Experiments in Economics: The Past, the Present,
and the Future"
Working Paper 14356(2008) 
National Bureau of Economic Research

\bibitem{name12} %website
Yann Algan, Pierre Cahuc,
\texttt{Social attitudes and economic development: an epidemiological approach}
VoxEU. Medium of publication
1/5/2017.
\texttt{<http://voxeu.org/article/trust-and-economic-development>.}

\bibitem{name13} %journal article
Jeffrey Butler,
''”The Right Amount of Trust"
Nota di Lavoro 61.(2010) 
Einaudi Institute for Economics and Finance, EIEF

\bibitem{name14} %journal article
Liu Xin, Datta Anwitaman, Rzadca Krzysztof,
''”Trust beyond reputaion: A compuatational
trust model based on stereotypes"
Electronic Commerce Research and Applications, Jan-Feb, Vol.12(1), p.24(2013) 
Science Direct

\bibitem{name15} %book
Claudio Cioffi-Revilla
\textit{Introduction to Computational Social Science -Principles and
Applications.}
New York, NY:
Springer London Heidelberg New York Dordrecht,
2014.
Springer.

\bibitem{name17} %journal article
Isaac Pinyol, Jordi, Sabater-Mir,
''”Computational trust and reputation models for open multi-agent systems: a review"
 Intell Rev 40:125 (2013)  
Springer Science+Business Media B.V.


%\bibitem{DBHS1}
%Alcosse, Howard
%''Diamond Bar High School."
%\textit[Internal Assessment: %Mathmatical Exploration}.
%Web.27 May 2015

%\bibitem{name1} %book
%Author(s)
%\textit{Title of Book.}
%City of Publication:
%Publisher,
%Year of Publication.
%Medium of Publication.


%\bibitem{name2} %magazine article
%Author(s)
%''Title of Article"
%Title of Periodical
%Day Month Year:
%pages.
%Medium of publication


%\bibitem{name3} %journal article
%Author(s)
%''Title of Article"
%Voloumn.Issue(Year)
%pages.
%Medium of publication


%\bibitem{name4} %website
%Editor,author, or compiler name(if available).
%\textit{Name of Site}
%Name of institution or organization affiliated with the site(sponser or publisher).
%date of resource creation(if available)
%Medium of publication.
%Date of access.
%\texttt{<http://www.daum.net>.}


%\bibitem{name5} %image
%Artiest's name,
%\textit{The Work of Art}
%Date of creation,
%Institution and city where the work is housed.
%textit{Name of Website},
%Medium of publication.
%Date of access.


\end{thebibliography}


\end{document}